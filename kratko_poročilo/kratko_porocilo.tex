\documentclass[a4paper, 11pt]{article}

\usepackage[slovene]{babel}
\usepackage[utf8]{inputenc}
\usepackage[T1]{fontenc}
\usepackage{lmodern}
\usepackage{amsmath}
\usepackage{amsfonts}
\usepackage{amssymb}
\usepackage{bbm}
\usepackage{hyperref}
\usepackage{makeidx}
\usepackage[most]{tcolorbox}

\definecolor{block-gray}{gray}{0.97}
\newtcolorbox{myquote}{colback=block-gray,
                        grow to right by=-2mm,
                        grow to left by=-2mm, 
                        boxrule=0pt, 
                        boxsep=0pt, 
                        breakable}

\title{\textbf{\LARGE{The firefighter problem}}}
\author{Klara Travnik in Karolina Šavli}
\date{16.\ november\ 2022}

\begin{document}

\maketitle

\section{Opis problema}

\noindent \emph{The firefighter problem} oziroma \emph{problem gasilca} obravnava širjenje in omejevanje požara
na grafu. Na začetku (v času $0$) požar izbruhne na nekem nizu točk in v vsakem časovnem
koraku se lahko na poljubno še nepogorelo oglišče postavi nov gasilec in požar omeji tako da lahko
napreduje zgolj na vozlišča, ki še niso zavarovana z gaslici in imajo za soseda vozlišče, ki gori.
Cilj problema je zajezitev požara, tako da čim več vozlišč ostane nepogorelih. 
Problem se lahko aplicira na mnoge probleme v realnem življenju, kot je 
na primer širjenje nalezljive bolezni. \\

\noindent IDEJA IN FORMALIZACIJA PROBLEMA \\
Podan imamo graf z vozlišči $V\left(G\right)$ in povezavami $E\left(G\right)$ ter 
fiksno število gasilcev $D$. \\
V času $t = 0$ požar izbruhne v nizu vozlišč $B_{init} \subseteq V$. Pogorela vozlišča
označimo kot \emph{burnt (b)}.
V času $t = 1$ se $D$ gasilcev postavi na še nepogorela vozlišča. Slednja vozlišča 
označimo kot \emph{defended (d)}.
V naslednjem časovnem koraku se lahko požar razširi zgolj na sosednja vozlišča, ki niso še \emph{defended}.
Za tem gasilci izbirajo vozlišča in proces se ponavlja dokler požar ni zajezen. \\
Za dan problem lahko zapišemo \textbf{celoštevilski linearni program (CLP)}, pri katerem
bomo maksimirali število nepogorelih vozlišč:


\begin{myquote}

    \begin{align*}
        & \text{max} \left|V\right| - \sum_{v \in V}{b_{v, T}} \\
        & \text{pri pogojih:} \\
        & b_{v,t} + d_{v,t} - b_{v',t-1} \ge 0 && \forall v \in V ,\, \forall v' \in N(v) ,\, \forall t \in \mathbb{N} \, 1 \le t \le T \\
        & b_{v,t} + d_{v,t} \le 1 && \forall v \in V ,\, \forall t \in \mathbb{N} ,\, 1 \le t \le T \\
        & b_{v,t} - b_{v,t-1} \ge 0 && \forall v \in B ,\, \forall t \in \mathbb{N} ,\, 1 \le t \le T \\
        & d_{v,t} - d_{v,t-1} \ge 0 && \forall v \in B ,\, \forall t \in \mathbb{N} ,\, 1 \le t \le T \\
        & \sum_{v \in V}{\left( d_{v,t} - d_{v, t-1} \right)} \le D && \forall t \in \mathbb{N} ,\, 1 \le t \le T \\
        & b_{v,0} = 1 && \forall v \in B_{init} \\
        & b_{v,0} = 0 && \forall v \in V \setminus B_{init} \\
        & d_{v,0} = 0 && \forall v \in V %\\
        % & b_{v,t} ,\, d_{v,t} \in \left\{ 0, 1 \right\} && \forall v \in V ,\, \forall t \in \mathbb{N} ,\, 1 \le t \le T \\
    \end{align*}

    $ b_{v,t} = \begin{cases}
        1, & \text{ če  vozlišče $v$ pogori v času $t$} ,\\
        0, & \text{ sicer.}
        \end{cases} $

    $ d_{v,t} = \begin{cases}
            1, & \text{ če je vozlišče $v$ rešeno v času $t$} ,\\
            0, & \text{ sicer.}
            \end{cases} $

    $$
        \forall v \in V ,\, \forall t \in \mathbb{N} ,\, 1 \le t \le T
    $$

\end{myquote}

\noindent Obrazložitev spremenljivk in korakov:
\begin{itemize}
    \item Imamo \textbf{celoštevilski spremenljivki} $b_{v,t}$ in $d_{v,t}$.
    \item Pogoj $$b_{v,t} + d_{v,t} - b_{v',t-1} \ge 0 \qquad \forall v \in V ,\, \forall v' \in N(v) ,\, \forall t \in \mathbb{N} \, 1 \le t \le T $$
    pove, da če je vozlišče $v'$ pogorelo v času $t-1$, za vse njegove sosede velja, da v času $t$  zagotovo pogorijo, če 
    niso rešeni. Če pa vozlišče $v'$ v času $t-1$ ni pogorelo, je še vseeno možno, da 
    vozlišče $v$ pogori (če kateri drugi sosedi zagorijo v času $t-1$), ali pa je rešeno.
    \item Pogoj $$b_{v,t} + d_{v,t} \le 1 \qquad \forall v \in V ,\, \forall t \in \mathbb{N} ,\, 1 \le t \le T$$
    pomeni, da vozlišče ne more biti hkrati rešeno in pogorelo.
    \item Pogoja $$b_{v,t} - b_{v,t-1} \ge 0 \qquad \forall v \in B ,\, \forall t \in \mathbb{N} ,\, 1 \le t \le T$$
    $$d_{v,t} - d_{v,t-1} \ge 0 \qquad \forall v \in B ,\, \forall t \in \mathbb{N} ,\, 1 \le t \le T$$
    povesta, da če je bilo vozlišče v nekem času $t$ rešeno ali je pogorelo, velja za rešeno oz. pogorelo tudi v vseh prihodnjih časih do $T$.
    \item Vsota $$\sum_{v \in V}{\left( d_{v,t} - d_{v, t-1} \right)} \le D \qquad \forall t \in \mathbb{N} ,\, 1 \le t \le T$$
    pogojuje število na novo rešenih vozlišč. V vsakem času $t \in  [1, T]$ je na voljo $D$ gasilcev, vsak lahko reši
    eno vozlišče, torej je v vsakem času največ $D$ rešenih. 
    \item  Začetni pogoji za spremenljivke pa povedo, da so vsa vozlišča iz množice vozlišč $B_{init}$, na 
    katerih se požar začne širiti, v času $0$ označena kot pogorela. 
    Vsa ostala vozlišča pa v času $0$ niso še niti pogorela niti rešena. \\
\end{itemize}

\section{Načrt dela}

\noindent V projektni nalogi bova predstavili \emph{The firefighter problem} s pomočjo
celoštevilskega linearnega programiranja. Algoritme bova implementirali
v programskem okolju \emph{Sage} in si po potrebi pomagali še s programskima
jezikoma \emph{Python} in \emph{R}. \\
Motivacijo in pomoč za delo bova/sva črplali s člankov ``García-Martínez, Blum, Rodríguez in Lozano''\cite{garcia2015} ter
``Fomin, Heggernes in van Leeuwen''\cite{fomin2015}. \\


% vi
\bibliographystyle{plain}
\bibliography{literatura}

\end{document}