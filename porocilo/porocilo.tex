\documentclass[a4paper, 12pt]{article}

\usepackage[slovene]{babel}
\usepackage[utf8]{inputenc}
\usepackage[T1]{fontenc}
\usepackage{lmodern}
\usepackage{amsmath}
\usepackage{amsfonts}
\usepackage{amssymb}
\usepackage{bbm}
\usepackage{hyperref}
\usepackage{makeidx}
\usepackage[most]{tcolorbox}
\usepackage{graphicx,subfig}


\textwidth 15cm
\textheight 24cm
\oddsidemargin.5cm
\evensidemargin.5cm
\topmargin-5mm
\addtolength{\footskip}{10pt}
\pagestyle{plain}
\overfullrule=15pt

% ================================================================================================

\begin{document}
    
\thispagestyle{empty}
\noindent{\large
Univerza v Ljubljani\\[2mm]
Fakulteta za matematiko in fiziko\\[2mm]
Finančna matematika 1.~stopnja}
\vfill

\begin{center}{\large
Karolina Šavli, Klara Travnik\\[5mm]
{\Huge \bf The firefighter problem}\\[5mm]
Projekt pri predmetu Finančni praktikum\\[1cm]}
\end{center}
\vfill

\noindent{\large Ljubljana, januar 2023}
\pagebreak

% ================================================================================================

\tableofcontents

\pagebreak

% ================================================================================================


\section{Naslov}
Besedilo ...

\subsection{Podnaslov}
Besedilo ...

\pagebreak

Opis in formulacija problema (opis, predstavitev clp-ja, koda clp-ja, ali_je_problem_končan, cas_potreben) [KLara]
Vizualizacija problema (koda barvanja, primer G2) [Karolina]
(Časovna zahtevnost algoritma): 
Testiranje programa glede na število vozlišč grafa, komentar grafov [Karolina]
Sklep in zaključek (uporaba problema gasilca v praksi (hiše, bolezen)) [Klara]




% ================================================================================================

% \bibliographystyle{plain}
% \bibliography{literatura}

% ================================================================================================

\end{document}