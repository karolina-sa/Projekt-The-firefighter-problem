\documentclass[a4paper, 12pt]{article}

\usepackage[slovene]{babel}
\usepackage[utf8]{inputenc}
\usepackage[T1]{fontenc}
\usepackage{lmodern}
\usepackage{amsmath}
\usepackage{amsfonts}
\usepackage{amssymb}
\usepackage{bbm}
\usepackage{hyperref}
\usepackage{makeidx}
\usepackage[most]{tcolorbox}
\usepackage{graphicx,subfig}


\textwidth 15cm
\textheight 24cm
\oddsidemargin.5cm
\evensidemargin.5cm
\topmargin-5mm
\addtolength{\footskip}{10pt}
\pagestyle{plain}
\overfullrule=15pt

% ================================================================================================

\begin{document}
    
\thispagestyle{empty}
\noindent{\large
Univerza v Ljubljani\\[2mm]
Fakulteta za matematiko in fiziko\\[2mm]
Finančna matematika 1.~stopnja}
\vfill

\begin{center}{\large
Karolina Šavli, Klara Travnik\\[5mm]
{\Huge \bf The firefighter problem}\\[5mm]
Projekt pri predmetu Finančni praktikum\\[1cm]}
\end{center}
\vfill

\noindent{\large Ljubljana, januar 2023}
\pagebreak

% ================================================================================================

\tableofcontents

\pagebreak

% ================================================================================================


\section{Naslov}
Besedilo ...

\subsection{Podnaslov}
\begin{itemize}
    \item Opis in formulacija problema (opis, predstavitev clp-ja, koda clp-ja, alijeproblemkončan, caspotreben) KLara
    \item Vizualizacija problema (koda barvanja, primer G2) Karolina
    \item (Časovna zahtevnost algoritma): 
    \item Testiranje programa glede na število vozlišč grafa, komentar grafov Karolina
    \item Sklep in zaključek (uporaba problema gasilca v praksi (hiše, bolezen)) Klara
\end{itemize}


\pagebreak

\section{Opis in formulacija problema}

\noindent \emph{The firefighter problem} oziroma \emph{problem gasilca} je optimizacijski problem, katerega cilj je 
minimiziranje števila pogorelih vozlišč na grafu. \\
Vhodni podatki problema so:
\begin{itemize}
    \item graf $G,$
    \item množica vozlišč $B_{init} \subseteq V\left(G\right)$, na katerih v času $0$ izbruhne požar,
    \item število gasilcev $D$.
\end{itemize} 
V vsaki časovni enoti ($t > 0$) gasilci izberejo nepogorela vozlišča, ki jih bodo rešili tako,
da čim bolj omejijo požar. Ta se razširi le na sosednja vozlišča pogorelih v prejšnji časovni enoti,
ki jih gasilci niso uspeli rešiti. Proces se ponavlja dokler požar ni zajezen.

Opisani problem sva v programu CoCalc, v programskem jeziku SageMaths 
zapisali kot \textbf{celoštevilski linearni program (CLP)}: \\

\begin{verbatim}
    def clp(G, B, gasilci):
    ''' vhodni podatki:
         G           izbran graf
         B           vozlišča, ki na začetku zgorijo
         gasilci     število gasilcev, ki v vsakem koraku gasijo požar
     izhodni podatki:
         seznam oblike [število časovnih enot, pogorela/burnt vozlišča po časih, 
         zaščitena/defended vozlišča po časih] '''
    cas = 10
    while True:
        casi = range(1, cas+1) # uprabljamo pri zankah
    
        # CLP:
        p = MixedIntegerLinearProgram(maximization=False) # CLP
        d = p.new_variable(binary=True) # spremenljivka, defended
        b = p.new_variable(binary=True) # spremenljivka, burnt

        p.set_objective(sum(b[i, cas] for i in G)) # minimiziramo število 
        pogorelih vozlišč na koncu 

        for t in casi:
            for i in G:
                for j in G[i]: # j je številka v seznamu vozlišča i, sosed od i
                    p.add_constraint(b[i,t] + d[i,t] - b[j,t-1] >= 0)
                p.add_constraint(b[i,t] + d[i,t] <= 1)
                p.add_constraint(b[i,t] - b[i,t-1] >= 0)
                p.add_constraint(d[i,t] - d[i,t-1] >= 0)
            p.add_constraint(sum((d[i,t] - d[i,t-1]) for i in G) <= gasilci)

        for i in G:
            p.add_constraint(b[i,0] == (1 if i in B else 0))
            p.add_constraint(d[i,0] == 0)
            
        k = p.solve()
        l = p.get_values(b)
        m = p.get_values(d)
        
        #Ali je problem končan?
        n = skrcitev(l, cas) # burnt vozlišča v cas
        e = skrcitev(m, cas) # defended vozlišča v cas
        skupaj = n + e
        
        koncan = 1
        # sosedi od pogorelih vozlišč so lahko pogoreli ali zaščiteni.
        Ne smejo biti prazna vozlišča
        for pogorelo_vozlisce in n:
            for sosed_od_pogorelo_vozlisce in G[pogorelo_vozlisce]:
                if sosed_od_pogorelo_vozlisce not in skupaj:
                    koncan = 0
        koncan
        
        if koncan == 1:
            break
        else:
            cas += 10
        
    return [k, l, m]
\end{verbatim}

CLP sva želeli zastaviti tako, da v argumentu funkcije ni potrebno nastaviti časa,
za katerega naj bi bil CLP končan. Zato sva nastavili nek začeten čas $cas = 10,$ za katerga 
je bil rešen algoritem. Potem sva preverili, če je ta \textbf{rešitev končna}, 
torej, ali je vrednost spremenljivk v zadnji časovni enoti ustrezna.
To pomeni, da za vsako vozlišče, ki je zgorelo, velja, da je vsako sosednje
vozlišče le-tega tudi zgorelo, ali pa bilo rešeno. V nasprotnem primeru 
proces še ne bi bil končen, in čas $cas$ se nastavi na večjo vrednost ter
ponovimo algoritem. \pagebreak

Za točen čas, za katerega dobimo končno rešitev (proces se v naslednjih časih ne spreminja),
sva napisali sledečo funkcijo:
\begin{verbatim}
def cas_potreben(G, B, gasilci):
''' iz p.solve() pridobi čas po katerem se nič več ne spremeni 
-> dobimo potreben čas '''
 #cas = 10 #začetni cas
cas = 10
while True:
    t, burnt, defended = clp(G, B, gasilci)

    urej_burnt = sorted(burnt.items(), key=lambda tup: tup[0][1])
    #uredi glede na čas po vozliščih naraščajoče
    urej_defended = sorted(defended.items(), key=lambda tup: tup[0][1]) 

    vredn_burnt= []
    for i, v in urej_burnt:
        vredn_burnt.append(v)
    # pridobim ven vrednosti spremnljivk b v časih in vozliščih naraščajoče

    vredn_defended= []
    for i, v in urej_defended:
        vredn_defended.append(v)
    # pridobim ven vrednosti spremnljivk d v časih in vozliščih naraščajoče

    # from itertools import islice
    from itertools import accumulate
    dolzina = [len(G)] * (cas +1) 
    #Vrednosti zgrupiram v paketke, v vsakem je toliko vrednosti, kolikor je vozlišč
    seznami_vrednosti_po_casih_burnt = [tuple(vredn_burnt[x - y: x]) 
    for x, y in zip(accumulate(dolzina), dolzina)]

    seznami_vrednosti_po_casih_defended = [tuple(vredn_defended[x - y: x]) 
    for x, y in zip(accumulate(dolzina), dolzina)]

    d = next(i for i in range(len(dolzina)) if all(len(set(l[i:i+2])) == 1 
    for l in (seznami_vrednosti_po_casih_burnt, seznami_vrednosti_po_casih_defended)))
        
    if d < cas:
        break
    else:
        cas += 10
return d
\end{verbatim}
Funkcija $cas\_potreben$ torej za vsak graf, podmnožico vozlišč $B$ grafa ter določeno število gasilcev
izračuna potreben čas za rešitev algoritma. Pomembna opomba tukaj je, da ta čas ni enak časovni zahtevnosti
algoritma. Potreben čas predstavlja število časovnih enot v procesu širjenja požara in reševanju vozlišč.


% ================================================================================================

% \bibliographystyle{plain}
% \bibliography{literatura}

% ================================================================================================

\end{document}