\documentclass[a4paper, 11pt]{article}

\usepackage[slovene]{babel}
\usepackage[utf8]{inputenc}
\usepackage[T1]{fontenc}
\usepackage{lmodern}
\usepackage{amsmath}
\usepackage{amsfonts}
\usepackage{amssymb}
\usepackage{bbm}
\usepackage{hyperref}
\usepackage{makeidx}

\title{\textbf{\LARGE{The firefighter problem}}}
\author{Klara Travnik in Karolina Šavli}
\date{16.\ november\ 2022}

\begin{document}

\maketitle

\noindent \textbf{OPIS PROBLEMA IN NAČRT DELA} \\

The firefighter problem oziroma problem gasilca obravnava širjenje in omejevanje požara
na grafu. Na začetku (v času $0$) požar izbruhne na nekem nizu točk. V vsakem časovnem
koraku se lahko na poljubno še nepogorelo oglišče postavi nov gasilec in požar omeji tako da lahko
napreduje zgolj na vozlišča, ki še niso zavarovana z gaslici in imajo za soseda vozlišče, ki gori.
Cilj problema je zajezitev požara, tako da čim več vozlišč ostane nepogorelih. 
Problem se lahko oplicira na mnoge probleme v realnem življenju, kot je 
na primer širjenje nalezljive bolezni. \\

\noindent FORMALIZACIJA PROBLEMA \\
Podan imamo graf z vozlišči $V\left(G\right)$ in povezavami $E\left(G\right)$.
Število vozlišč in povezav označimo z $n = \left|V\left(G\right)\right|$ in $m = \left|E\left(G\right)\right|$.
Dano imamo tudi fiksno število gasilcev in sicer $D$. \\
V času $t = 0$ požar izbruhne v nizu vozlišč $B_{init} \subseteq V$. Pogorela vozlišča
označimo kot \emph{burnt}.
V času $t = 1$ se $D$ gasilcev postavi na še nepogorela vozlišča. Slednja vozlišča 
označimo kot \emph{defended}.
V naslednjem časovnem koraku se lahko požar razširi zgolj na sosednja vozlišča, ki niso še \emph{defended}.
Za tem gasilci izbirajo vozlišča in proces se ponavlja dokler požar ni zajezen. \\
Za dan problem lahko zapišemo celoštevilki linearni program (CLP), pri katerem
bomo maksimirali število nepogorelih vozlišč:
\begin{align*}
    & \text{max} \left|V\right| - \sum_{v \in V}{b_{v, T}} \\
    & \text{pri pogojih:} \\
    & b_{v,t} + d_{v,t} - b_{v',t-1} \ge 0 && \forall v \in V ,\, \forall v' \in N(v) ,\, \forall t \in \mathbb{N} \, 1 \le t \le T \\
    & b_{v,t} + d_{v,t} \le 1 && \forall v \in V ,\, \forall t \in \mathbb{N} ,\, 1 \le t \le T \\
    & b_{v,t} - b_{v,t-1} \ge 0 && \forall v \in B ,\, \forall t \in \mathbb{N} ,\, 1 \le t \le T \\
    & d_{v,t} - d_{v,t-1} \ge 0 && \forall v \in B ,\, \forall t \in \mathbb{N} ,\, 1 \le t \le T \\
    & \sum_{v \in V}{\left( d_{v,t} - d_{v, t-1} \right)} \le D && \forall t \in \mathbb{N} ,\, 1 \le t \le T \\
    & b_{v,0} = 1 && \forall v \in B_{init} \\
    & b_{v,0} = 0 && \forall v \in V \setminus B_{init} \\
    & d_{v,0} = 0 && \forall v \in V \\
    & b_{v,t} ,\, d_{v,t} \in \left\{ 0, 1 \right\} && \forall v \in V ,\, \forall t \in \mathbb{N} ,\, 1 \le t \le T \\
\end{align*}

Obrazložitev spremenljivk in korakov:

\begin{itemize} 
    \item $b_{v,t} = \begin{cases}
    1, & \text{ če  vozlišče $v$ pogori v času $t$} ,\\
    0, & \text{ sicer.}
    \end{cases} $

    \item $d_{v,t} = \begin{cases}
        1, & \text{ če je vozlišče $v$ rešeno v času $t$} ,\\
        0, & \text{ sicer.}
        \end{cases} $
\end{itemize}
Imamo torej \textbf{celoštevilski spremenljivki.} \\
Če je vozlišče $v$ pogorelo v času $t$, je tudi v vseh prihodnjih časih do časa $T$ označeno kot pogorelo. 
Podobno tudi za vozlišča, ki jih gasilci rešijo; če je rešeno v času $t$, ostane rešeno do konca.

Pogoj $$b_{v,t} + d_{v,t} - b_{v',t-1} \ge 0 \qquad \forall v \in V ,\, \forall v' \in N(v) ,\, \forall t \in \mathbb{N} \, 1 \le t \le T $$
pove, da če je vozlišče $v'$ pogorelo v času $t-1$, za vse njegove sosede velja, da v času $t$  zagotovo pogorijo, če 
niso rešeni. Če pa vozlišče $v'$ v času $t-1$ ni pogorelo, je še vseeno možno, da je vozlišče $v$ pogori (če kateri drugi sosedi zagorijo v času $t-1$) ali pa je rešeno.

Pogoj $$b_{v,t} + d_{v,t} \le 1 \qquad \forall v \in V ,\, \forall t \in \mathbb{N} ,\, 1 \le t \le T$$
pomeni, da ne more vozlišče biti hkrati rešeno in pogorelo.

Pogoja $$b_{v,t} - b_{v,t-1} \ge 0 \qquad \forall v \in B ,\, \forall t \in \mathbb{N} ,\, 1 \le t \le T$$ in 
$$d_{v,t} - d_{v,t-1} \ge 0 \qquad \forall v \in B ,\, \forall t \in \mathbb{N} ,\, 1 \le t \le T$$
povesta, da če je bilo vozlišče v nekem času $t$ rešeno ali je pogorelo, je rešeno oz. pogorelo tudi v vseh prihodnjih časih.








    













\noindent \textbf{VIRI} \\

\bibliographystyle{plain}
\bibliography{literatura.bib}

\printindex

\end{document}